% This file is part of the TargetSelection project
% Copyright 2019 the authors.

% To-dos
% ------
% - make a to-do list
% - make style notes

% Style notes
% - 

\documentclass[modern]{aastex62}
% \usepackage{graphicx, xcolor}
% \usepackage[sort&compress]{natbib}

% units macros
\newcommand{\unit}[1]{\mathrm{#1}}
\newcommand{\pc}{\unit{pc}}
\newcommand{\kpc}{\unit{kpc}}

% math macros
\newcommand{\dd}{\mathrm{d}}
\newcommand{\T}{^{\mathsf{T}}}
\newcommand{\given}{\,|\,}

% chemical macros
\newcommand{\abundance}[2]{\mathrm{\left[{#1}/{#2}\right]}}
\newcommand{\alphafe}{\abundance{\alpha}{Fe}}

% text macros
\newcommand{\documentname}{\textsl{Article}}
\newcommand{\sectionname}{Section}
\newcommand{\code}[1]{\texttt{\detokenize{#1}}}
\newcommand{\acronym}[1]{{\small{#1}}}
\newcommand{\project}[1]{\textsl{#1}}
\newcommand{\foreign}[1]{\textsl{#1}}
\newcommand{\etal}{\foreign{et al.}}

% margins and headers and spacing changes
\setlength{\parindent}{1.4em} % trust in Hogg
\addtolength{\topmargin}{-0.5in}
\addtolength{\textheight}{1.0in}
\shorttitle{desiderata for target selection}
\shortauthors{hogg & rix}

\begin{document}\sloppy\sloppypar\raggedbottom\frenchspacing % trust in Hogg
% \graphicspath{ {figures/} }
%\DeclareGraphicsExtensions{.pdf,.eps,.png}

\title{\textbf{%
Target selection for spectroscopic surveys:\\
General requirements for statistical and legacy value%
}}

\author[0000-0003-2866-9403]{David W. Hogg}
\affil{Center for Cosmology and Particle Physics, Department of Physics, New York University, 726 Broadway, New York, NY 10003, USA}
\affil{Center for Data Science, New York University, 60 Fifth Ave, New York, NY 10011, USA}
\affil{Max-Planck-Institut f\"ur Astronomie, K\"onigstuhl 17, D-69117 Heidelberg}
\affil{Flatiron Institute, 162 Fifth Ave, New York, NY 10010, USA}

\author[0000-0003-4996-9069]{Hans-Walter Rix}
\affil{Max-Planck-Institut f\"ur Astronomie, K\"onigstuhl 17, D-69117 Heidelberg}

\begin{abstract}\noindent
% context
We are in a magical time for spectroscopic surveys; the astronomical
community is building and operating large projects in the areas of
large-scale structure, Milky Way cartography, stellar physics,
exoplanet discovery, and explosive transients.
These projects all have in common that they must pre-select sources
for spectroscopic observation, from prior imaging (and spectroscopic)
data.
They also have in common that they want their results to be useful for
statistics, population inferences, and legacy use in future projects.
% aims
Here we lay out some general principles for target selection that, if
obeyed, will help a survey to achieve this latter goal.
% methods
The key idea, from which all principles flow, is that the survey must
be able to produce a computationally tractable and (in appropriate ways)
accurate likelihood function for parameters of interest, given the
data obtained.
% results
Among our findings and recommendations are the following:
You want to base your selection as much as possible on quantities that
can be reliably and straightforwardly predicted by the relevant
physical models.
You want to avoid having the selection of one source
depend strongly on the properties of \emph{other} sources.
You don't need to know everything about how sources were selected, you
just need to know those aspects of selection that project most strongly
onto the matters of great interest.
\end{abstract}

\keywords{\raggedright
 foo
 ---
 bar
}

\section{Introduction}

foo bar whatever

\acknowledgements It is a pleasure to thank the \project{Milky Way
  Mapper} Collaboration for help with refining these principles.

\facilities{
ESA \project{Gaia},
}

\software{
foo,
bar
}

\begin{thebibliography}{}
\bibitem[De Silva \etal(2015)]{galah} De Silva, G.~M., Freeman, K.~C., Bland-Hawthorn, J., \etal\ 2015, \mnras, 449, 2604 
\end{thebibliography}

\end{document}
